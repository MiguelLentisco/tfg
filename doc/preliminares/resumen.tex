% !TeX root = ../libro.tex
% !TeX encoding = utf8
%
%*******************************************************
% Resumen
%*******************************************************

% \manualmark
% \markboth{\textsc{Introducción}}{\textsc{Introducción}}

\chapter*{Resumen}\label{ch:resumen}
%\addcontentsline{toc}{chapter}{Resumen}

Este trabajo está dirigido principalmente a la investigación de problemas poco tratados e investigados, donde aún queda mucho por explorar, en el área de series temporales con modelos de redes neuronales LSTM. Nos centraremos en tres problemas bien diferenciados: selección de modelos, detección de anomalías y predicción de series discretas. Antes de abordarlos, veremos una introducción para tener un entendimiento básico del funcionamiento de las redes neuronales y concretamente de las redes LSTM. Después repasaremos los conceptos sobre series temporales más generales que serán el dominio de nuestros problemas. Finalmente acabamos nuestros conceptos previos con las métricas que usaremos para valorar los modelos utilizados. Ahora sí pasaremos a los problemas: en la selección de modelos investigaremos una nueva heurística frente a la validación clásica y discutiremos su efectividad. Para la detección de anomalías ofrecemos un detector sencillo pero funcional junto a unos métodos para crear instancias anómalas a partir de las normales. Finalmente estudiaremos la predicción de series temporales discretas obteniendo éstas de series continuas mediante métodos de discretización.

\paragraph{PALABRAS CLAVE}
\begin{itemize*}[label=,itemsep=4em,itemjoin=\hspace{2em}]
  \item redes neuronales
  \item LSTM
  \item series temporales
  \item selección de modelos
  \item validación
  \item hiperparámetros
  \item detección de anomalías
  \item detector
  \item perturbación
  \item series temporales discretas
  \item predicción
  \item discretización
\end{itemize*}

\endinput
