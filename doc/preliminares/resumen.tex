% !TeX root = ../libro.tex
% !TeX encoding = utf8
%
%*******************************************************
% Resumen
%*******************************************************

% \manualmark
% \markboth{\textsc{Introducción}}{\textsc{Introducción}}

\chapter*{Resumen}\label{ch:resumen}
%\addcontentsline{toc}{chapter}{Resumen}

Este trabajo está dirigido principalmente a la investigación de problemas en el área de series temporales con modelos de redes neuronales LSTM que han sido poco tratados e investigados y donde existe una gran cantidad de trabajo por explorar. Nos centraremos en dos problemas bien diferenciados: selección de modelos para clasificación de series temporales y detección de anomalías en series temporales. Antes de abordarlos, veremos una introducción para tener un entendimiento básico del funcionamiento de las redes neuronales y concretamente de las redes LSTM. A continuación, repasaremos el marco teórico general del análisis de series temporales que serán el dominio de nuestros problemas. También se aborda brevemente las métricas que usaremos para valorar los modelos utilizados. Continuamos desarrollando las partes fundamentales de la investigación, empezando con la selección de modelos donde analizaremos una nueva heurística llamada \emph{Perturbation Validation} frente a la validación clásica y discutiremos su efectividad mediante la experimentación en un gran conjunto de datos de series con diversos modelos, incluyendo uno basado en LSTM. Para la parte de detección de anomalías, ofrecemos un detector sencillo pero funcional junto a unos métodos para crear instancias anómalas a partir de las normales, que utilizaremos para validar el rendimiento del detector que hemos realizado a la vez que comprobamos la utilidad de estos métodos de alteración de series.

\paragraph{PALABRAS CLAVE:}
\begin{itemize*}[label=,itemsep=1em,itemjoin=\hspace{1em}]
  \item redes neuronales
  \item LSTM
  \item series temporales
  \item selección de modelos
  \item validación
  \item selección de hiperparámetros
  \item detección de anomalías
  \item detector
  \item perturbación
\end{itemize*}

\endinput
