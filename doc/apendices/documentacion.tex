% !TeX root = ../libro.tex
% !TeX encoding = utf8

\chapter{Documentación}\label{ap:documentacion}

En este apéndice se deja la documentación en estilo \emph{python} de la documentación de las distintas clases, métodos y funciones más importantes implementados en el proyecto.

\section{Selección de Modelos}

Las clases implementadas para la parte de selección de modelos que se pueden encontrar en la carpeta $PV/src$.

\subsection{Perturbated Validation}

Esta clase y sus métodos se encuentran en el archivo $PV.py$.

\paragraph{PV}

Clase PV que implementa el método para calcular y manejar la heurística PV. Guarda los datos de las series originales, las perturbaciones realizadas, los ratio de error, el nombre del dataset que se está perturbando, y valores auxiliares para imprimir gráficas del cálculo del PV.

\begin{lstlisting}
class PV:
    """
        Clase que implementa Perturbation Validation (PV).

        Attributes
        ----------
        X : np.array
            Dataset
        y : np.array
            Conjuto de etiquetas perturbadas
        ds_name : str
            Nombre del dataset
        errs : np.array
            Errores tomados
        counter : int
            Contador auxiliar
        fig : Figure
            Figura actual
        ax : Axes
            Ejes actuales
    """
\end{lstlisting}

\paragraph{Constructor}

Constructor de la clase PV que necesita los datos originales, el número de perturbaciones, el nombre del \emph{dataset}, y el inicio y fin de los ratio de error. Crea los conjuntos de etiquetas perturbadas.

\begin{lstlisting}
def __init__(self, X, y, n_pv = 5, ds_name = "", err_ini = 0.1,
                 err_fin = 0.3):
        """
            Inicializa la clase creando las etiquetas perturbadas.

            Las perturbaciones se realizan tomando un %err de cada
            clase, poniendole otra etiqueta distinta.

            Se toman "n_pv" puntos entre [err_ini, err_fin].

            Parameters
            ----------
            X : np.array
                Dataset
            y : np.array
                Etiquetas
            n_pv: int
                Número de puntos/errores
            ds_name: str
                Nombre del datases
            err_ini : float
                Error inicial
            err_fin : float
                Error final
        """
\end{lstlisting}

\paragraph{Cálculo PV}

Método para calcular el valor PV de un modelo dado.

\begin{lstlisting}
def get_pv(self, clf, clf_name = "", plot = True):
        """
            Calcula el PV score para el clasificador.

            Parameters
            ----------
            clf : Classifier
                Clasificador
            clf_name : str
                Nombre del clasificador

            Returns
            -------
            pv : float
                PV score
            accs : list(float)
                accs obtenidos
        """
\end{lstlisting}

\paragraph{Dibujar cálculo PV}

Método para representar en una gráfica los valores de la métrica $acc$ obtenidos en el cálculo de PV junto a la recta de regresión obtenida.

\begin{lstlisting}
def plot_pv(self, errs, accs, poly, pv, clf_name = ""):
        """
            Dibuja los puntos y la recta de regresión en la figura actual.

            Parameters
            ----------
            errs : np.array
                Errores
            accs : np.array
                acc obtenidos
            poly : np.array
                Recta de regresión
            pv : float
                Valor PV
            clf_name : str
                Nombre del clasificador
        """
\end{lstlisting}

\paragraph{Guardar gráfica}

Método para guardar en una imagen .png el gráfico del método $plot\_pv$.

\begin{lstlisting}
def save_graph(self, name_fig):
        """
            Guarda el gráfico de los resultados en un .png

            Parameters
            ----------
            name_fig : str
                Nombre (ruta) de la imagen a guardar.
        """
\end{lstlisting}

\subsection{Clasificador LSTM}

Esta clase y sus métodos se encuentran en el archivo $LSTM.py$.

\paragraph{LSTM}

La clase LSTM que implementa el clasificador LSTM. Guarda el modelo LSTM, el número de clases, la longitud de las series, opciones de entrenamiento y para gráficas de entrenamiento.

\begin{lstlisting}
class LSTM(BaseEstimator):
    """
        Implementación de una red neuronal con capas LSTM.

        Attributes
        ----------
        counter : int, static
            Valor auxiliar para ruta de imagen
        model : Sequential
            Modelo red neuronal
        history : list
            Historial del entrenamiento
        n_clases : int
            Número de clases de las etiquetas
        input_shape : tuple
            Forma de los datos
        epochs: int
            Número de épocas para entrenamiento
        verbose : int
            Información sobre el entrenamiento
        save_hist : boolean
            Si guardar las gráficas de los entrenamientos
    """
\end{lstlisting}

\paragraph{Constructor}

Constructor de la clase LSTM que guarda opciones de entrenamiento.

\begin{lstlisting}
def __init__(self, epochs, n_neurs = 80, verbose = 0, save_hist = False,
             n_clases = -1):
        """
            Inicializamos la red LSTM.

            Attributes
            ----------
            epochs : int
                Número de épocas para entrenamiento
            n_neurs : int
                Número de neuronas LSTM
            verbose : int
                Información sobre el entrenamiento
            save_hist : boolean
                Si guardar las gráficas de los entrenamientos
            n_clases : int
                Número de clases a predecir
        """
\end{lstlisting}

\paragraph{Creación del modelo}

Método para crear el modelo LSTM.

\begin{lstlisting}
def create_model(self):
        """
            Crea el modelo LSTM.
        """
\end{lstlisting}

\paragraph{Compilar el modelo}

Compila el modelo con el optimizador ADAM y la función de pérdida entropía cruzada categórica.

\begin{lstlisting}
def compile_model(self):
        """
            Compila el modelo con optimizador ADAM y función de pérdida
            categorical_crossentropy.
        """
\end{lstlisting}

\paragraph{Entrenamiento}

Método para entrenar el modelo con el conjunto de datos, con las épocas guardadas, validación al 10\% y con parada temprana.

\begin{lstlisting}
def fit(self, X, y):
        """
            Entrenamos el modelo.

            Parameters
            ----------
            X : numpy.array
                Datos de entrenamiento
            y : numpy.array
                Etiquetas de entrenamiento
        """
\end{lstlisting}

\paragraph{Cálculo métrica}

Método para calcular la métrica $acc$ en el conjunto de datos pasado.

\begin{lstlisting}
def score(self, X, y):
        """
            Calcula el acc con los datos que se le pasan.

            Parameters
            ----------
            X : numpy.array
                Datos test
            y : numpy.array
                Etiquetas test

            Returns
            ----------
            acc : float
                accuracy obtenida
        """
\end{lstlisting}

\paragraph{Guardar gráfica de entrenamiento}

Método para guardar el historial de entrenamiento en una imagen.

\begin{lstlisting}
def save_history(self):
        """
            Guarda el historial en una imagen.
        """
\end{lstlisting}

\subsection{Clasificadores}

Los clasificadores adicionales que usamos para comparar modelos, comparten dos métodos generales: entrenamiento y cálculo de la métrica.

\paragraph{Entrenamiento}

Método que se encarga de entrenar el modelo usando una muestra de datos.

\begin{lstlisting}
def fit(self, X, y):
    """
        Entrena el modelo.

        Parameters
        ----------
        X : numpy.array
            Datos de entrenamiento
        y : numpy.array
            Etiquetas de entrenamiento
    """
\end{lstlisting}

\subparagraph{Cálculo de métrica}

Método que se encarga de calcular la métrica ($accuracy$) de un modelo en un conjunto de datos.

\begin{lstlisting}
def score(self, X, y):
    """
        Calcula el acc con los datos que se le pasan.

        Parameters
        ----------
        X : numpy.array
            Datos test
        y : numpy.array
            Etiquetas test

        Returns
        ----------
        acc : float
            accuracy obtenida
    """
\end{lstlisting}

\subsubsection{C4.5}

Esta clase y sus métodos se encuentran en el archivo $RClassifiers.py$.

\paragraph{C45}

La clase C45 que usa el árbol de decisión C4.5 que guarda el modelo.

\begin{lstlisting}
class C45(BaseEstimator):
    """
        Implementa el árbol de decisión C4.5.

        Attributes
        ----------
        model : clasificador en R
            El clasificador (clase en R)
    """
\end{lstlisting}

\subsubsection{C5.0}

Esta clase y sus métodos se encuentran en el archivo $RClassifiers.py$.

\paragraph{C50}

La clase C50 que usa el árbol de decisión C5.0 que guarda el modelo, y también el valor del \emph{boosting}.

\begin{lstlisting}
class C50(BaseEstimator):
    """
        Implementa el árbol de decisión C5.0 (con boosting o no).

        Attributes
        ----------
        model : clasificador en R
            El clasificador (clase en R)
        boosting : int
            El valor del boosting
    """
\end{lstlisting}

\paragraph{Constructor}

Constructor de la clase C50 que se le pasa el número de \emph{boosting} que se necesite.

\begin{lstlisting}
def __init__(self, boosting = 10):
        """
            Inicializa el clasificador.

            Parameters
            ----------
            boosting : int
                El valor del boosting
        """
\end{lstlisting}

\subsubsection{Recursive Partioning Tree}

Esta clase y sus métodos se encuentran en el archivo $RClassifiers.py$.

\paragraph{RPart}

Clase que usa el árbol RPart.

\begin{lstlisting}
class RPart(BaseEstimator):
    """
        Implementa el árbol de decisión RPart (Recursive Partioning Tree).

        Attributes
        ----------
        model : clasificador en R
            El clasificador (clase en R)
    """
\end{lstlisting}

\subsubsection{Condicional Tree}

Esta clase y sus métodos se encuentran en el archivo $RClassifiers.py$.

\paragraph{CTree}

La clase CTree implementa el uso del árbol de decisión Condicional Tree.

\begin{lstlisting}
class CTree(BaseEstimator):
    """
        Implementa el árbol de decisión CTree (Conditional Inference Tree).

        Attributes
        ----------
        model : clasificador en R
            El clasificador (clase en R)
    """
\end{lstlisting}


\subsubsection{$k$-NN}

Esta clase y sus métodos se encuentran en el archivo $KNN.py$.


\paragraph{Clase KNN}

Clase que implementa el clasificador $k$-NN que se le puede pasar el $k$ fijo o que lo calcule automáticamente tomado como la raíz cuadrada del número de datos.

\begin{lstlisting}
class KNN(BaseEstimator):
    """
        Implementa el clasificador KNN (K-Nearest neighbors).

        Parameters
        ----------
        k : int
            Número de vecinos
        model : KNeighborsClassifier
            Modelo k-NN
        metric : str, metric
            Métrica que usar con KNN
        n_jobs : int
            Número de procesadores usados
    """
\end{lstlisting}

\paragraph{Constructor}

Constructor de la clase KNN que necesita el número de vecinos, la métrica y el número de procesadores.

\begin{lstlisting}
def __init__(self, k = None, metric = "euclidean", n_jobs = 1):
        """
            Inicializa el clasificador.

            Parameters
            ----------
            k : int
                Número de vecinos
            metric : str, metric
                Métrica que usar con KNN
            n_jobs : int
                Número de procesadores
        """
\end{lstlisting}

\subsubsection{$k$-NN + DTW}

Esta clase y sus métodos se encuentran en el archivo $RClassifiers.py$.

\paragraph{DTW}

Clase que implementa el clasificador $k$-NN con métrica DTW, que guarda los datos de entrenamiento, el número de vecinos y el tamaño de la ventana para aplicar DTW.

\begin{lstlisting}
class DTW(BaseEstimator):
    """
        Clase que implementa K-Nearest Neighbors con la distancia DTW
        usando la implementación del paquete "IncDTW".

        Attributes
        ----------
        data : R.DataFrame
            Datos transformados en un objeto dataframe de R
        k : int
            Número de vecinos
        window_shift : int
            Tamaño de la ventana para aplicar DTW
    """
\end{lstlisting}

\paragraph{Constructor}

Constructor de la clase DTW que necesita el número de vecinos y el tamaño de la ventana para el cálculo de la métrica DTW.

\begin{lstlisting}
def __init__(self, k = 1, window_shift = 5):
    """
        Constructor de la clase, debe hacerse solo una vez por dataset.

        Parameters
        ----------
        k : int
            Números de vecinos
        window_shift : int
            Tamaño de la ventana para aplicar DTW
    """
\end{lstlisting}

\section{Detección de anomalías}

Funciones y clases relativas a la parte de detección de anomalías que se encuentran en la carpeta $AD/src$.

\subsection{Alteración de series}

Funciones para la creación de anomalías en base a las series normales implementadas en el archivo $alteraciones.py$.

\paragraph{Tramo aleatorio}

Función para escoger un tramo aleatorio de la serie en función a la longitud indicada (máxima, mínima, fija).

\begin{lstlisting}
def random_slice(x, max_length = None, min_length = None,
                   length = None, pos = None, border = 0):
    """
        Se encarga de elegir un tramo aleatorio de una serie que queda
        determinado por una posición y longitud, de manera que el tramo
        elegido es [posición, posición + longitud).

        Se puede determinar una longitud máxima o mínima, o incluso
        especificar una longitud o posición fijada. También se puede
        indicar si excluir los extremos (añadir borde).

        Parameters
        ----------
        x : np.numpy
            Serie temporal que alterar
        max_length : int, None
            Longitud máxima de la perturbación
        min_length : int, None
            Longitud mínima de la perturbación
        length : int, None
            Longitud fija de la perturbación
        pos : int, None
            Posición fija de la perturbación
        border : int
            Borde para excluir la perturbación

        Returns
        -------
        pos : int
            Posición de la perturbación
        length : int
            Longitud de la perturbación
    """
\end{lstlisting}

\paragraph{Ruido gaussiano}

Método para alterar un tramo aleatorio de la serie añadiendo ruido gaussiano mediante un parámetro $\sigma$ que controla la intensidad de esta perturbación, y la longitud máxima y mínima de esta.

\begin{lstlisting}
def gaussian_noise(x, max_length, min_length = 3, std = 3, neg = False,
                   border = 0, neg_random = True):
    """
        Crea una perturbación de ruido gaussiano añadiendo en un
        tramo aleatorio un muestreo de la función de densidad normal.
        Se puede controlar la intensidad.

        Además se puede activar aleatoriamente (50%) o de manera fija que la
        alteración gaussiana sea negativa.

        Parameters
        ----------
        x : np.numpy
            La serie para alterar
        max_length : int
            Longitud máxima de la alteración
        min_length : int
            Longitud minima de la alteración
        std : float
            Controla la intensidad de la alteración
        neg : boolean
            Si invertir la señal gaussiana
        border : int
            El borde para excluir la perturbación
        neg_random : boolean
            Si se invierte aleatoriamente las señales

        Returns
        -------
        x : np.numpy
            Una copia de la señal perturbada
    """
\end{lstlisting}

\paragraph{Pulso sinusoidal-gaussiano}

Método para alterar un tramo aleatorio de la serie añadiendo un pulso sinusoidal-gaussiano mediante su frecuencia $fc$, un parámetro $\sigma$ que controla la intensidad de la perturbación y la longitud máxima y mínima de esta.

\begin{lstlisting}
def gaussian_sine_pulse(x, max_length, min_length = 3, fc = 1.5, std = 3,
                        border = 0):
    """
        Crea una perturbación con un pulso sinusoidal-gaussiano añadido en un
        tramo aleatorio. Se puede controlar la intensidad y la frecuencia
        del pulso.

        Parameters
        ----------
        x : np.numpy
            La serie para alterar
        max_length : int
            Longitud máxima de la alteración
        min_length : int
            Longitud minima de la alteración
        fc : float
            Frecuencia de la señal del pulso
        std : float
            Controla la intensidad de la alteración
        border : int
            El borde para excluir la perturbación

        Returns
        -------
        x : np.numpy
            Una copia de la señal perturbada
    """
\end{lstlisting}

\paragraph{Estacionalidad}

Método para alterar un tramo aleatorio de la serie modificando la estacionalidad de la descomposición STL (dada con un periodo) por un parámetro $\sigma$ que controla la intensidad y la longitud máxima y mínima de esta.

\begin{lstlisting}
def modify_season(x, period, max_length, min_length = 3, std = 1, border = 0):
    """
        Crea una perturbación multiplicando por un real la estacionalidad
        de un tramo aleatorio de la serie. Se necesita el periodo para
        realizar la descomposición STL.

        Parameters
        ----------
        x : np.numpy
            La serie para alterar
        period : int
            Periodo de repetición de la serie para descomposición STL
        max_length : int
            Longitud máxima de la alteración
        min_length : int
            Longitud minima de la alteración
        std : float
            Controla la intensidad de la alteración
        border : int
            El borde para excluir la perturbación
    """
\end{lstlisting}

\paragraph{Tendencia}

Método para alterar un tramo aleatorio de la serie modificando la tendencia de la descomposición STL (dada con un periodo) por un parámetro $\sigma$ que controla la intensidad y la longitud máxima y mínima de esta.

\begin{lstlisting}
def modify_trend(x, period, max_length, min_length = 3, std = 1, border = 0):
    """
        Crea una perturbación multiplicando por un real la tendencia
        de un tramo aleatorio de la serie. Se necesita el periodo para
        realizar la descomposición STL.

        Parameters
        ----------
        x : np.numpy
            La serie para alterar
        period : int
            Periodo de repetición de la serie para descomposición STL
        max_length : int
            Longitud máxima de la alteración
        min_length : int
            Longitud minima de la alteración
        std : float
            Controla la intensidad de la alteración
        border : int
            El borde para excluir la perturbación
    """
\end{lstlisting}

\subsection{Detector}

Clase y sus métodos implementados para crear el detector de anomalías, implementado en $detector.py$

\paragraph{Clase LSTM\_AD}

Clase que implementa el detector de anomalías basado en autoencoder LSTM. Mantiene el modelo LSTM, la probabilidad estimada y otros parámetros de entrenamiento.

\begin{lstlisting}
class LSTM_AD:
    """
        Clase que implementa un detector de anomalías usando
        un modelo autoencoder con capas LSTM.

        Attributes
        ----------
        model : keras.Sequential
            Autoencoder LSTM
        n_neur : int
            Número de neuronas base para las capas
        alpha : float
            Parámetro de regularización L2
        lr : float
            Learning rate
        epochs : int
            Número de épocas de entrenamiento
        mode : int
            Si incluir espacio de codificación (1) o no (2)
        hist : keras.Historial
            Historial de entrenamiento
        kernel : scipy.gaussian_kde
            Distribución de errores estimada
    """
\end{lstlisting}

\paragraph{Constructor}

Constructor de la clase LSTM\_AD que guarda los parámetros relativos al entrenamiento y al modo de arquitectura.

\begin{lstlisting}
def __init__(self, n_neur = 32, alpha = 0, lr = 0.001, epochs = 300,
             mode = 2):
    """
        Constructor de la clase

        Parameters
        ----------
        n_neur : int
            Número de neuronas base para las capas
        alpha : float
            Parámetro de regularización L2
        lr : float
            Learning rate
        epochs : int
            Número de épocas de entrenamiento
        mode : int
            Si incluir espacio de codificación (1) o no (2)
    """
\end{lstlisting}

\paragraph{Creación del modelo}

Función para crear la arquitectura del modelo autoencoder LSTM.

\begin{lstlisting}
def create_model(self, X):
    """
        Crea la arquitectura del autoencoder LSTM con los atributos
        de la clase.

        Parameters
        ----------
        X : np.numpy
            Series temporales
    """
\end{lstlisting}

\paragraph{Compilación}

Función para compilar el modelo autoencoder LSTM.

\begin{lstlisting}
def compile_model(self):
    """
        Compila el modelo con ADAM añadiendo un clip de 1, learning
        rate especificado y minimizando el error cuadrático medio.
    """
\end{lstlisting}

\paragraph{Entrenamiento}

Función para entrenar el modelo autoencoder LSTM.

\begin{lstlisting}
def load_model(self, path):
    """
        Carga el modelo de unos pesos guardados en un archivo

        Parameters
        ----------
        path : str
            Ruta donde está el archivo de los pesos
    """
\end{lstlisting}

\paragraph{Reconstrucción}

Función para obtener las reconstrucciones de un conjunto de series temporales.

\begin{lstlisting}
"""
    Obtiene las reconstrucciones del autoencoder para las series.

    Parameters
    ----------
    X : numpy.array
        Datasets de series temporales

    Returns
    -------
    reconstrucciones : numpy.array
        Reconstrucciones de las series temporales
"""
\end{lstlisting}

\paragraph{Estimar distribución}

Función para estimar la distribución de los errores de reconstrucción.

\begin{lstlisting}
def fit_kernel(self, X):
    """
        Ajustamos la distribución de los errores de reconstrucción
        con los datos de entrenamiento.

        Parameters
        ----------
        X : numpy.array
            Dataset de series temporales
    """
\end{lstlisting}

\paragraph{Calcular probabilidades}

Función para obtener las probabilidades de ser serie anómala para un conjunto de series temporales.

\begin{lstlisting}
def predict_prob(self, X):
    """
        Devolvemos las probabilidades de ser serie anómala para
        cada serie del dataset

        Parameters
        ----------
        X : numpy.array
            Dataset de series temporales

        Returns
        -------
        probs : numpy.array
            Probabilidades de anomalía para cada serie
    """
\end{lstlisting}

\subsection{Cálculo Curva Precision-Recall}

Las funciones para calcular la métrica $AUC$-$PR$ (curva precisión-recall) que se encuentran en el archivo $calc\_pr.py$.

\paragraph{Contar anomalías}

Se cuentan el número de anomalías detectadas en función de las probabilidades de las series de ser anómalas y de un umbral de probabilidad al partir del cual se considera que es anómala.

\begin{lstlisting}
def count_anomalies(probs, threshold):
    """
        Cuenta cuantas anomalías hay en función a la probabilidad de serlo
        y un umbral de probabilidad.

        Parameters
        ----------
        probs : np.numpy
            Array con probabilidades de cada serie de ser anómala
        threshold : float
            Umbral de probabilidad a partir del cual se considera anómala

        Returns
        -------
        n_anomalies : int
            Número de anomalías detectadas
    """
\end{lstlisting}

\paragraph{Calcular sensibilidad}

Se calcula la sensibilidad del modelo en base a las probabilidades de las series anómalas y un umbral.

\begin{lstlisting}
def calc_recall(probs_anomalies, threshold):
    """
        Calcula la sensibilidad (recall) de un modelo en base a las
        probabilidades de las series anómalas.

        Parameters
        ----------
        probs_anomalies : np.numpy
            Array con probabilidades de anomalías de las series anómalas
        threshold : float
            Umbral de probabilidad

        Returns
        -------
        recall : float
            Sensibilidad del modelo
    """
\end{lstlisting}

\paragraph{Calcular precisión}

Se calcula la precisión del modelo en base a las probabilidades de las series anómalas y normales junto a un umbral.

\begin{lstlisting}
def calc_precision(probs_normal, probs_anomalies, threshold):
    """
        Calcula la precisión de un modelo en base a las probabilidades
        de las series anómalas y normales.

        Parameters
        ----------
        probs_normal : np.numpy
            Array con probabilidades anomalías de las series normales
        probs_anomalies : np.numpy
            Array con probabilidades anomalías de las series anómalas
        threshold : float
            Umbral de probabilidad

        Returns
        -------
        precision : float
            Precisión del modelo
    """
\end{lstlisting}

\paragraph{Curva Precision-Recall}

Se calcula la métrica $PR$ tomando el área debajo de la curva Precision-Recall integrando en el cuadrado $[0, 1]^2$. Además se imprime una figura mostrando la curva que se forma.

\begin{lstlisting}
def recall_precision_curve(X_normal, X_anomalies, model, clf_name = "clf",
                           title = "recall-precision curve", axis = None,
                           plot = True):
    """
        Calcula la métrica PR y además muestra la curva Precision-Recall
        del modelo.

        Parameters
        ----------
        X_normal : np.numpy
            Series normales
        X_anomalies : np.numpy
            Series anómalas
        model : detector
            Detector de anomalías
        clf_name : str
            Nombre del detector
        title : str
            Título de la gráfica
        axis : matplotlib.axis
            Objeto para imprimir las gráficas
        plot : boolean
            Si imprimir cosas opcionales de la gráfica

        Returns
        -------
        pr_score : float
            Valor de la métrica PR
    """
\end{lstlisting}

\section{Predicción de series discretas}

Las clases implementadas para la parte de predicción de series discretas que se pueden encontrar en la carpeta $SD/src$.

\subsection{Métodos de discretización}

Implementación del método de discretización SAX y codificación de las palabras en valores numéricos en el archivo $discretization.py$.

\subsubsection{Método SAX}

\paragraph{SAX}

El método SAX implementado en la clase SAX, que incluye como método auxiliar el método PAA. Incluye todos los métodos para los que realiza el algoritmo y otros adicionales para poder ser usados con la biblioteca \emph{sklearn}.

\begin{lstlisting}
class SAX:
    """
        Clase que implementa el método de discretiación SAX.

        Attributes
        ----------
        tam_window : int
            Tamaño de ventana.
        breakpoints : list
            Lista de los puntos de ruptura.
        alphabet : list
            Lista del alfabeto.
        mean : np.array
            Medias de las series.
        std : np.array
            Desviaciones estándar de las series.
    """
\end{lstlisting}

\paragraph{Constructor}

Constructor de la clase SAX que necesita únicamente el tamaño de ventana para el método PAA y el tamaño del alfabeto.

\begin{lstlisting}
def __init__(self, tam_window, alphabet_tam):
        """
            Constructor.

            Parameters
            ----------
            tam_window : int
                Tamaño de ventana.
            alphabet_tam : int
                Tamaño del alfabeto.
        """
\end{lstlisting}

\paragraph{Normalización}

Normaliza todas las series temporales que se le pasan, aunque se tiene que haber llamado primero al método \emph{fit} para poder tener las medias y las desviaciones estándar.

\begin{lstlisting}
def standarization(self, X):
    """
        Normaliza las series temporales que se pasan.

        Parameters
        ----------
        X : np.array
            Las series temporales.

        Returns
        -------
        res : np.array
            Las series temporales normalizadas.
    """
\end{lstlisting}

\paragraph{PAA}

Realiza el método de reducción de dimensión PAA a la serie que se le pasa, tomando la media de cada ventana.

\begin{lstlisting}
def PAA(self, x):
        """
            Realiza la reducción de dimensión PAA.

            Parameters
            ----------
            x : np.array
                La serie temporal.

            Returns
            -------
            res : np.array
                La serie temporal reducida.
        """
\end{lstlisting}

\paragraph{Cálculo de puntos de ruptura}

Calcula los puntos de ruptura para el método SAX. Son los bordes de los intervalos que dividen el área bajo la distribución normal en partes iguales de valor $1 / k$, siendo $k$ el tamaño del alfabeto.

\begin{lstlisting}
def calc_breakpoints(self, alphabet_tam):
        """
            Cálculo de los puntos de ruptura, de manera que entre cada par
            de valores consecutivos hay 1/alphabet_tam de área debajo
            de la curva de una distribución normal.

            Parameters
            ----------
            alphabet_tam : int
                Tamaño del alfabeto.

            Returns
            -------
            bp : list
                Lista con los puntos de ruptura.
        """
\end{lstlisting}

\paragraph{Tranformación en palabra}

Transforma la serie, previamente normalizada y transformada con PAA, en una palabra del alfabeto usando los puntos de ruptura.

\begin{lstlisting}
def string_transformation(self, x_paa):
        """
            Transforma una serie en una palabra tomando para cada valor
            si se encuentra en el intervalo determinado por los puntos
            de ruptura.

            Parameters
            ----------
            x_paa : np.array
                Serie temporal.

            Returns
            -------
            word : list
                Serie convertida en palabra (lista de carácteres)
        """
\end{lstlisting}

\paragraph{Discretización}

Aplica PAA junto a la discretización en palabra a una serie previamente normalizada.

\begin{lstlisting}
"""
           Discretiza aplicando PAA y la transformación a palabra.

           Parameters
           ----------
           x : np.array
               Serie temporal.

           Returns
           -------
           word : list
               Serie transformada en palabra.
       """
\end{lstlisting}

\paragraph{Ajustar}

Método para \emph{sklearn} que ajusta las medias y las desviaciones típicas de las series.

\begin{lstlisting}
def fit(self, X, y = None):
        """
            Ajusta las medias y desviaciones típicas de las series.

            Parameters
            ----------
            X : np.array
                Series temporales
            y : None
                Por compatibilidad.

            Returns
            -------
            self : SAX
                El propio objeto.
        """
\end{lstlisting}

\paragraph{Transformar}

Método para \emph{sklearn} que transforma las series en palabras usando SAX. Se debe usar antes el método \emph{fit}.

\begin{lstlisting}
def transform(self, X, y = None):
        """
            Realiza la transformación SAX a todas las series que se le pasan.

            Parameters
            ----------
            X : np.array
                Series temporales.
            y : None
                Por compatibilidad.

            Returns
            -------
            X_strings : list
                Lista con las series transformadas en series.
        """
\end{lstlisting}

\paragraph{Ajustar y transformar}

Método para \emph{sklearn} que ajusta las series y las transforma a la vez.

\begin{lstlisting}
def fit_transform(self, X, y = None):
        """
            Realiza la transformación SAX a todas las series que se le pasan.

            Parameters
            ----------
            X : np.array
                Series temporales.
            y : None
                Por compatibilidad.

            Returns
            -------
            X_strings : list
                Lista con las series transformadas en series.
        """
\end{lstlisting}

\paragraph{Transformación inversa}

Método para \emph{sklearn} por compatibilidad, por razones obvias no se puede realizar una transformación inversa por lo que aplica la identidad.

\begin{lstlisting}
def inverse_transform(self, X):
        """
            Por compatibilidad, deshace la transformación.

            Parameters
            ----------
            X : np.array
                Palabras discretizadas.

            Returns
            X : np.array
                Palabras discretizadas.
        """
\end{lstlisting}

\subsubsection{Codificador de palabras}

\paragraph{StringEncoder}

Clase auxiliar implementada para poder codificar las palabras en variables numéricas, en concreto etiquetas naturales, para su uso en los predictores. Se puede realizar la transformación inversa para obtener las palabras originales.

\begin{lstlisting}
class StringEncoder:
    """
        Clase para codificar las series

        Attributes
        ----------
        encoders : list
            Lista con los codificadores para cada serie.
    """
\end{lstlisting}

\paragraph{Ajuste}

Ajusta las listas de palabras con los codificadores para transformar cada letra en una etiqueta numérica.

\begin{lstlisting}
def fit(self, X, y = None):
        """
            Ajusta los codificadores a las palabras.

            Parameters
            ----------
            X : list
                Lista de palabras.
            y : None
                Por compatibilidad.

            Returns
            -------
            self : StringEncoder
                El objeto de la clase.
        """
\end{lstlisting}

\paragraph{Transformación}

Transforma la lista de palabras en las etiquetas codificadas previamente ajustadas con el método \emph{fit}.

\begin{lstlisting}
def transform(self, X, y = None):
        """
            Transforma las palabras en etiquetas numéricas.

            Parameters
            ----------
            X : list
                Lista de palabras.
            y : None
                Por compatibilidad.

            Returns
            -------
            res : list
                Lista de las palabras transformadas en etiquetas.
        """
\end{lstlisting}

\paragraph{Ajuste y transformación}

Realiza el ajuste de los codificadores y la transformación a la vez.

\begin{lstlisting}
def fit_transform(self, X, y = None):
       """
           Transforma las palabras en etiquetas numéricas.

           Parameters
           ----------
           X : list
               Lista de palabras.
           y : None
               Por compatibilidad.

           Returns
           -------
           res : list
               Lista de etiquetas.
       """
\end{lstlisting}

\paragraph{Transformación inversa}

Deshace la codificación de las etiquetas, reconstruyendo las etiquetas originales.

\begin{lstlisting}
def inverse_transform(self, X):
        """
            Transforma de vuelta las etiquetas en palabras.

            Parameters
            ----------
            X : list
                Lista de etiquetas.

            Returns
            -------
            res : list
                Lista de palabras.
        """
\end{lstlisting}
\endinput
%------------------------------------------------------------------------------------
% FIN DEL APÉNDICE.
%------------------------------------------------------------------------------------
