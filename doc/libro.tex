% !TEX program = pdflatex
% !TEX encoding = UTF-8 Unicode

% Plantilla de la clase `scrbook` del paquete KOMA-script para la
% elaboración de un TFG siguiendo las directrices del la comisión del
% Grado en Matemáticas de la Universidad de Granada.

% Francisco Torralbo Torralbo
% miércoles, 29 de abril de 2020

\documentclass{scrbook}

\KOMAoptions{%
  fontsize=10pt,        % Tamaño de fuente
  paper=a4,             % Tamaño del papel
  headings=normal,      % Tamaño de letra para los títulos: small, normal, big
  % parskip=half,         % Espacio entre párrafos: full (una línea) o half (media línea)
  headsepline=false,    % Una linea separa la cabecera del texto
  cleardoublepage=empty,% No imprime cabecera ni pie en páginas en blanco
  chapterprefix=false,  % No antepone el texto "capítulo" antes del número
  appendixprefix=false,	% No antepone el texto "Apéndice" antes de la letra
  listof=totoc,		    	% Añade a la tabla de contenidos la lista de tablas y figuras
  index=totoc,			    % Añade a la talba de contenidos una entrada para el índice
  bibliography=totoc,	  % Añade a la tabla de contenidos una entrada para bibliografía
  BCOR=5mm,           % Reserva margen interior para la encuadernación.
                        % El valor dependerá el tipo de encuadernado y del grosor del libro.
  DIV=10,             % Cálcula el diseño de página según ciertos
                        % parámetros. Al aumentar el número aumentamos el ancho de texto y disminuimos el ancho del margen. Una opción de 14 producirá márgenes estrechos y texto ancho.
}

% INFORMACIÓN PARA LA VERSIÓN IMPRESA
% Si el documento ha de ser impreso en papel de tamaño a4 pero el tamaño del documento (elegido en \KOMAoptions con la ocpión paper) no es a4 descomentar la línea que carga el paquete `crop` más abajo. El paquete crop se encargará de centrar el documento en un a4 e imprimir unas guías de corte. El procedimiento completo para imprenta sería el siguiente:
% 0. Determinar, según el tipo de encuadernación del documento, el ancho reservado para el proceso de encuadernación (preguntar en la imprenta), es decir, la anchura del área del papel que se pierde durante el proceso de encuadernación. Fijar la varibale BCOR de \KOMAoptions a dicho valor.
% 1. Descomentar la siguiente línea e imprimir una única página con las guías de corte
% 2. Cambiar la opción `cross` por `cam` (o `off`) en el paquete crop y volver a compilar. Imprimir el documento (las guías de corte impresas no inferfieren con el texto).
% 3. Usar la página con las guías impresas en el punto 1 para cortar todas las páginas.

% \usepackage[a4, odd, center, pdflatex, cross]{crop} % Permite imprimir el documento en un a4 (si el tamaño es más pequeño) mostrando unas guías de corte. Útil para imprenta.

% VERSIÓN ELECTRÓNICA PARA TABLETA
% Las opciones siguientes seleccionan un tamaño de impresión similar a una tableta de 9 pulgadas con márgenes estrechos. Útil para producir una versión en pdf para ser leída en una tableta en lugar de impresa.
% Para que la portada quede centrada correctamente hay que editar el
% archivo `portada.tex` y eliminar el entorno `addmargin`

% \KOMAoptions{fontsize=10pt, paper=19.7104cm:14.7828cm, twoside=false, BCOR=0cm, DIV=14}

% ---------------------------------------------------------------------
%	PAQUETES
% ---------------------------------------------------------------------

% CODIFICACIÓN E IDIOMA
% ---------------------------------------------------------------------
\usepackage[utf8]{inputenc} 			    % Codificación de caracteres

% Selección del idioma: cargamos por defecto inglés y español (aunque este último es el idioma por defecto para el documento). Cuando queramos cambiar de idioma escribiremos:
% \selectlanguage{english} o \selectlanguage{spanish}

\usepackage[english, spanish, es-nodecimaldot, es-noindentfirst, es-tabla]{babel}

% Opciones cargadas para el paquete babel:
  % es-nodecimaldot: No cambia el punto decimal por una coma en modo matemático.
  % es-noindentfirst: No sangra los párrafos tras los títulos.
  % es-tabla: cambia el título del entorno `table` de "Cuadro" a "Tabla"

% Otras opciones del paquete spanish-babel:
  \unaccentedoperators % Desactiva los acentos en los operadores matemáticso (p.e. \lim, \max, ...). Eliminar esta opción si queremos que vayan acentuados

% MATEMÁTICAS
% ---------------------------------------------------------------------
\usepackage{amsmath, amsthm, amssymb} % Paquetes matemáticas
\usepackage{mathtools}                % Añade mejoras a amsmath
\mathtoolsset{showonlyrefs=true}      % sólo se numeran las ecuaciones que se usan
\usepackage[mathscr]{eucal} 					% Proporciona el comando \mathscr para
                                      % fuentes de tipo manuscrito en modo matemático sin sobreescribir el comando \mathcal

% TIPOGRAFÍA
% ---------------------------------------------------------------------
% El paquete microtype mejora la tipografía del documento.
\usepackage[activate={true,nocompatibility},final,tracking=true,kerning=true,spacing=true,factor=1100,stretch=10,shrink=10]{microtype}

% Las tipografías elegidas para el documento son las siguientes
% Normal font: 			URW Palladio typeface.
% Sans-serif font: 	Iwona
% Monospace font: 	Inconsolata
% Consultar http://www.tug.dk/FontCatalogue/ para seleccionar otra tipografía.
% Es conveniente elegir aquellas que tienen soporte matemático.
\usepackage[T1]{fontenc}
\usepackage[sc, osf]{mathpazo} \linespread{1.05}
\usepackage[scaled=.95,type1]{cabin} % sans serif in style of Gill Sans
\usepackage{inconsolata}
% \renewcommand{\sfdefault}{iwona}


% Selecciona el tipo de fuente para los títulos (capítulo, sección, subsección) del documento.
\setkomafont{disposition}{\sffamily\bfseries}

% Cambia el ancho de la cita. Al inicio de un capítulo podemos usar el comando \dictum[autor]{cita} para añadir una cita famosa de un autor.
\renewcommand{\dictumwidth}{0.45\textwidth}

\recalctypearea % Necesario tras definir la tipografía a usar.

% TABLAS, GRÁFICOS Y LISTADOS DE CÓDIGO
% ---------------------------------------------------------------------
\usepackage{booktabs}
% \renewcommand{\arraystretch}{1.5} % Aumenta el espacio vertical entre las filas de un entorno tabular

\usepackage{xcolor, graphicx}
% Carpeta donde buscar los archivos de imagen por defecto
\graphicspath{{img/}}

% IMAGEN DE LA PORTADA
% Existen varias opciones para la imagen de fondo de la portada del TFG. Todas ellas tienen en logotipo de la universidad de Granada en la cabecera. Las opciones son las siguientes:
% 1. portada-ugr y portada-ugr-color: diseño con marca de agua basada en el logo de la UGR (en escala de grises y color).
% 2. portada-ugr-sencilla y portada-ugr-sencilla-color: portada únicamente con el logotipo de la UGR en la cabecera.
\usepackage{eso-pic}
\newcommand\BackgroundPic{%
	\put(0,0){%
		\parbox[b][\paperheight]{\paperwidth}{%
			\vfill
			\centering
      % Indicar la imagen de fondo en el siguiente comando
			\includegraphics[width=\paperwidth,height=\paperheight,%
			keepaspectratio]{portada-ugr-sencilla-color}%
			\vfill
}}}

\usepackage{listings} % Para la inclusión de trozos de código

% CABECERAS
% ---------------------------------------------------------------------
% Si queremos modificar las cabeceras del documento podemos usar el paquete
% `scrlayer-scrpage` de KOMA-Script. Consultar la documentación al respecto.
% \usepackage[automark]{scrlayer-scrpage}

% VARIOS
% ---------------------------------------------------------------------

%\usepackage{showkeys}	% Muestra las etiquetas del documento. Útil para revisar las referencias cruzadas.

% ÍNDICE
% Para generar el índice hay que compilar el documento con MakeIndex. Generalmente los editores se encargan de ello automáticamente.
% ----------------------------------------------------------------------
% \index{} para añadir un elemento
% \index{main!sub} para añadir un elementos "sub" bajo la categoría "main".
% \index{termino|textbf} para dar formato al número de página (negrita).
% \index{termino|see{termino relacionado}} para crear una referencia cruzada

% Ejemplo: \index{espacio homogéneo}, \index{superficie!mínima}, \index{esfera|see{espacio homogéneo}}
\usepackage{makeidx}
%\usepackage{showidx} % Muestra en el margen del documento las entradas añadidas al índice. Útil para revisar el documento. Si está activo el índice no se genera
\makeindex

% ---------------------------------------------------------------------
% COMANDOS Y ENTORNOS
% ---------------------------------------------------------------------
% Cargamos un archivo externo donde hemos incluido todos los comandos
% propios que vamos a usar en el documento.
\input{paquetes/comandos-entornos.tex}

% --------------------------------------------------------------------
% INFORMACIÓN DEL TFG Y EL AUTOR
% --------------------------------------------------------------------
\usepackage{xspace} % Para problemas de espaciado al definir comandos

\newcommand{\miTitulo}{Análisis y modelado de series temporales con métodos
estadísticos basados en Deep Learning\xspace}
\newcommand{\miNombre}{Miguel Lentisco Ballesteros\xspace}
\newcommand{\miGrado}{Doble Grado en Ingeniería Informática y Matemáticas}
\newcommand{\miFacultad}{Escuela Técnica Superior de Ingenierías Informática y de Telecomunicación \\ Facultad de Ciencias}
\newcommand{\miUniversidad}{Universidad de Granada}
% Añadir tantos tutores como sea necesario separando cada uno de ellos
% mediante el comando `\\\medskip` y una línea en blanco
\newcommand{\miTutor}{
  José Manuel Benítez Sánchez \\ \emph{Ciencias de la Computación e Inteligencia Artificial}
  \\\medskip

  Miguel Lastra Leidinger \\ \emph{Lenguajes y Sistemas Informáticos}
}
\newcommand{\miCurso}{2019-2020\xspace}

% HYPERREFERENCES
% --------------------------------------------------------------------
\usepackage{xurl}
\usepackage[pagebackref]{hyperref}
\input{paquetes/hyperref}

\begin{document}

% --------------------------------------------------------------------
% FRONTMATTER
% -------------------------------------------------------------------
%\frontmatter % Desactiva la numeración de capítulos y usa numeración romana para las páginas

% \pagestyle{plain} % No imprime cabeceras

\include{preliminares/portada}
\include{preliminares/titulo}
\include{preliminares/declaracion-originalidad}
% !TeX root = ../libro.tex
% !TeX encoding = utf8
%
%*******************************************************
% Resumen
%*******************************************************

% \manualmark
% \markboth{\textsc{Introducción}}{\textsc{Introducción}}

\chapter*{Resumen}\label{ch:resumen}
%\addcontentsline{toc}{chapter}{Resumen}

Este trabajo está dirigido principalmente a la investigación de problemas en el área de series temporales con modelos de redes neuronales LSTM que han sido poco tratados e investigados y donde existe una gran cantidad de trabajo por explorar. Nos centraremos en dos problemas bien diferenciados: selección de modelos para clasificación de series temporales y detección de anomalías en series temporales. Antes de abordarlos, veremos una introducción para tener un entendimiento básico del funcionamiento de las redes neuronales y concretamente de las redes LSTM. Después repasaremos el marco teórico general del análisis de series temporales que serán el dominio de nuestros problemas. Finalmente abordaremos las métricas que usaremos para valorar los modelos utilizados. Respecto a los distintos problemas que hemos investigado en este trabajo: en la selección de modelos investigaremos una nueva heurística frente a la validación clásica y discutiremos su efectividad mediante la experimentación en un gran conjunto de datos de series con diversos modelos, incluyendo uno basado en LSTM. Para la detección de anomalías ofrecemos un detector sencillo pero funcional junto a unos métodos para crear instancias anómalas a partir de las normales, que utilizaremos para validar el rendimiento del detector que hemos realizado a la vez que comprobamos la utilidad de estos métodos de alteración de series.

\paragraph{PALABRAS CLAVE:}
\begin{itemize*}[label=,itemsep=1em,itemjoin=\hspace{1em}]
  \item redes neuronales
  \item LSTM
  \item series temporales
  \item selección de modelos
  \item validación
  \item selección de hiperparámetros
  \item detección de anomalías
  \item detector
  \item perturbación
\end{itemize*}

\endinput

% !TeX root = ../libro.tex
% !TeX encoding = utf8
%
%*******************************************************
% Summary
%*******************************************************

\selectlanguage{english}
\chapter*{Summary}\label{ch:summary}
%\addcontentsline{toc}{chapter}{Summar}

The main goal of this project is to tackle research problems where there has been little research on, focusing on the domain of time series and solving these problems using models based on LSTM neural networks architectures.

First, we present a brief background of the project with an introduction to its basis, the motivation that led to its realization, objectives to accomplish and the structure of this document.

The first part is formed by the basic concepts used in the development of this project: Deep Learning, time series and metrics. The basics of Deep Learning starts with a short tour through the neural networks history with the aim to learn about its origins, leading to the most basic but functional neural network: the feed-forward neural network. We explain in detail how this model work, allowing us to understand the majority of neural networks models used nowadays as these networks share the fundamentals to a large degree of similarity. Finally, we list a few of the most well-known neural network architectures, focusing on the LSTM cells. This type of neural network tries to learn certain time-dependent patterns in the data, being perfectly suited for dealing with time series. This goal is accomplished through the use of information obtained of the past inputs that flows into the neuron like a feedback connection, providing it with a context that can be utilized for a better prediction of the current input.

Regarding the time series, we present the main results of the current research on the time series analysis with its mathematical foundation. First, we explore the probabilistic theory that includes the stochastic processes, successions of random variables, and the stationary processes, which are processes whose distribution does not change with a temporal shift. Then we study the lineal models like ARMA that are used for time series analysis, and its applications with the time series decomposition (like STL decomposition) or the time series \emph{differencing}. Combining either of these methods with the ARMA model leads us into the ARIMA and SARIMA models. We also introduce and explain a few techniques related to the process of time series \emph{discretization} such as the SAX method or a method based on the $k$-means clustering algorithm.

In the last section of this part we make a quick review of the metrics that we are going to use to validate the models developed in the project: for classification problems the accuracy, F-score and the precision-recall curves and for regression the usual error metrics like mean squared error or mean absolute error.

The first part of this project addresses the problem of model selection in classification tasks, that is classically done with the cross-validation and the hold-out-test validation. We study a new heuristic called \emph{Perturbation Validation} that does not relay on a train/test partition as the classical selection, but introduces tiny perturbations into the labels of the dataset and measures how the model responses to these changes. We study and explain in depth how it works, what it measures, how we interpret its value and how can it be useful. In order to confirm this, we experiment with several machine learning models, including a LSTM neural network, in a big repository of diverse time series datasets with both model and hyper-parameters selection. We analyze in detail the results to prove the effectiveness of this heuristic, showing that it can contribute to the model selection problem.

In the second and last part we address the anomaly detection problem: detecting atypical behavior in time series. We discuss the big obstacles in this type of problems: the lack of datasets with labeled data, mainly used in the training of models based on supervised learning; and the models developed being too specific for the task they are solving thus they can not be deployed in other areas. We provide a simple but useful model based on LSTM architecture that can detect several types of anomalies, provided a well-known pattern in the time series. We also present various methods to modify the normal series in order to create artificially samples of anomalies that can be used, for example, to validate our model. We evaluate the detector and the methods using them with both a real and synthetic dataset, analyzing the performance of the detector and the quality of the perturbations.

Also, we have included two appendices regarding the documentation of the main classes, methods and functions; and diverse information about the implemented software, such as the languages (mostly \emph{Python}) and the packages used.

\paragraph{KEYWORDS:}
\begin{itemize*}[label=,itemsep=1em,itemjoin=\hspace{1em}]
  \item neural networks
  \item LSTM
  \item time series
  \item model selection
  \item validation
  \item hyper-parameters selection
  \item anomaly detection
  \item detector
  \item perturbation
\end{itemize*}

% Al finalizar el resumen en inglés, volvemos a seleccionar el idioma español para el documento
\selectlanguage{spanish}
\endinput

%\include{preliminares/dedicatoria}                % Opcional
\include{preliminares/tablacontenidos}
            % Opcional

% \pagestyle{scrheadings} % A partir de ahora sí imprime cabeceras

% !TeX root = ../libro.tex
% !TeX encoding = utf8
%
%*******************************************************
% Introducción
%*******************************************************

% \manualmark
% \markboth{\textsc{Introducción}}{\textsc{Introducción}}

\chapter{Introducción}\label{ch:introduccion}

En las dos últimas décadas, gracias al uso de las GPUs \cite{oh2004gpu} se ha hecho posible un uso potente y extendido del \textbf{aprendizaje profundo} (\textit{deep learning}), una de las subáreas del \textbf{aprendizaje automático} (\emph{machine learning}) más conocidas y usadas actualmente debido a su gran versatilidad de aplicación en diversos tipos de problemas, su fácil diseño y su buen rendimiento. Este campo intenta resolver problemas muy diversos que no tienen una solución fácil de diseñar \textbf{manualmente} (reconocimiento en imágenes, conducción autónoma, domótica...), mediante el uso de las \textbf{redes neuronales} (\emph{Neural Networks}) \cite{landahl1943statistical, hebb1949organization}: algoritmos inspirados en el sistema neuronal del cerebro que, como el propio nombre indica, se estructuran mediante capas formadas por \emph{neuronas} que están \emph{conectadas} entre sí formando así una red compleja donde se propaga la información de una capa a otra.

Dentro de las muy variadas y complejas arquitecturas y tipos de capas que se han ido desarrollando destacamos las \textbf{redes neuronales recurrentes} (\emph{recurrent neural networks}, RNN) \cite{hopfield1982neural, jordan1997serial}, redes neuronales que introducen una novedad: la extracción y guardado de información de una entrada en un instante determinado ($x_t$) para su uso posterior ($x_{t+1}$); en otras palabras, la red utiliza no solo la entrada/dato actual sino también tiene en cuenta \textbf{relaciones de las entradas anteriores}, creándose de esta manera una especie de dependencia temporal.

Este tipo de redes se ha utilizado mucho en tareas donde existe esta dependencia temporal, donde destacan mayormente las \textbf{series temporales} (sucesiones de datos ordenados por tiempo) \cite{wang2017origin} donde existen una gran variedad de problemas como por ejemplo: predicción de valores (mercados, nº de pacientes, texto predictivo), clasificación de series (electrocardiogramas) o detección de anomalías (sensores, tráfico servidores).

La más conocida dentro de este tipo de arquitecturas es la \emph{Long Short Term Memory} (LSTM) \cite{hochreiter1997long} que intenta solventar un gran problema dentro de las RNN: la pérdida de dependencias temporales \textbf{a largo plazo}. Implementa un tipo especial de \emph{neurona} llamada \textbf{célula de memoria} para solucionar esto, permitiendo aprender u olvidar información del pasado según convenga \cite{wang2017origin}.

\section{Motivación}

En este campo tan extenso, existen muchos problemas que se han tratado poco o que son muy recientes y todavía no se han investigado; de aquí surge nuestra idea de ahondar en estos dominios de problemas de series temporales tan poco explorados. Nos centraremos así en dar soluciones a estos problemas mediante arquitecturas de redes neuronales LSTM para aprovechar la especialización en cuanto a la extración y uso de información de patrones y dependencias temporales que generalmente existen en las series.

\section{Objetivos}

Los objetivos fundamentales se centrarán en el estudio, investigación y obtención de posibles resultados en problemas pocos tratados actualmente mediante el uso de técnicas basados en \emph{Deep Learning}, concretamente modelos de redes neuronales con arquitecturas que utilizan capas LSTM.

Trataremos tres problemas diversos, cada uno desarrollados en una parte indpendiente:

\begin{enumerate}
  \item Selección de modelos (\autoref{part:pv}).
  \item Detección de anomalías (\autoref{part:ad}).
  \item Predicción de series temporales discretas (\autoref{part:sd}).
\end{enumerate}

\section{Estructura}

Comenzamos con una introducción en \autoref{ch:introduccion} donde desarrollamos brevemente el contexto del trabajo, junto con la motivación, los objetivos a alcanzar y la estructura del documento.

Hacemos un repaso de los conceptos previos más relevantes para poder entender este trabajo: sobre el aprendizaje profundo en \autoref{ch:dl}, series temporales en \autoref{ch:st} y las métricas utilizadas para valorar nuestros modelos en \autoref{ch:metricas},

Después estudiamos en cada parte el problema concreto que hemos ido desarrollando: en \autoref{part:pv} abordamos el problema de selección de modelos, investigando una nueva propuesta de métrica de validación de modelos llamada $PV$ (\emph{Perturbated Validation}) que estudiamos frente a la validación clásica.

En \autoref{part:ad} nos centramos en la detección de anomalías, modelando una detector capaz de clasificar series anómalas y también una serie de métodos capaces de alterar las series normales para crear otras anómalas.

La última parte en \autoref{part:sd} valoramos el comportamiento de las redes neuronales en series temporales discretas, un ámbito poco frecuente de uso con redes neuronales.

Se han incluido además dos apéndices con la documentación más relevante de las clases y funciones implementadas (\autoref{ap:documentacion}) y otro con información sobre el software desarrollado (\autoref{ap:software}).

\endinput



% --------------------------------------------------------------------
% MAINMATTER
% --------------------------------------------------------------------
%\mainmatter % activa la numeración de capítulos, resetea la numeración de las páginas y usa números arábigos
\setpartpreamble[c][0.75\linewidth]{
	\bigskip % Deja un espacio vertical en la parte superior
  Hacemos un repaso de conceptos básicos para tener los conocimientos necesarios a la hora de entender los trabajos y resultados del proyecto. En \autoref{ch:dl} desarrollamos la parte del aprendizaje profundo: se proporciona una breve introducción histórica, el funcionamiento básico de las redes neuronales, las arquitecturas más importantes utilizadas actualmente y el modelo concreto LSTM que usaremos en nuestros modelos. En \autoref{ch:st} presentamos la teoría matemática detrás de las series temporales, junto con el estudio y desarrollo necesario para poder trabajar y analizarlas. Finalmente en \autoref{ch:metricas} vemos rápidamente las métricas: funciones que usaremos para valorar el rendimiento de nuestros modelos.
}

\part{Conceptos previos}\label{part:conceptos-previos}
\include{conceptos-previos/aprendizaje-profundo}
% !TeX root = ../libro.tex
% !TeX encoding = utf8
%
%*******************************************************
% Series Temporales
%*******************************************************

\chapter{Series Temporales}\label{ch:st}

[EN CONSTRUCCIÓN]

\subsection{Descomposición STL}

En las series se pueden observar ciertos componentes comunes con patrones diferenciados en los que se pueden descomponer para hacer un mejor análisis y manuplacion de estas \cite{hyndman2018forecasting}:

\begin{enumerate}
  \item \textbf{Tendencia}: patrones de crecimiento o disminución a largo plazo.
  \item \textbf{Estacionalidad}: patrones afectados por factores estacionales (un día concreto del año, una estación como verano) que suelen tener una frecuencia fija.
  \item \textbf{Ciclo}: fluctuaciones que no parece tener una frecuencia fija, es decir, que son efímeros.
\end{enumerate}

Estas descomposiciones suelen realizarse en las componentes mencionadas: la \textbf{tendencia-cíclica} $S_t$ (se suele decir solamente tendencia), la \textbf{estacionalidad} $T_t$ y los \textbf{restos} $R_t$ (lo que sobra). Además podemos considerar un modelo aditivo \eqref{eq:aditivo} donde se suman las componentes o multiplicativo \eqref{eq:multiplicativo}; donde se multiplican \cite{hyndman2018forecasting}.

\begin{equation}
  y_t = S_t + T_t + R_t
  \label{eq:aditivo}
\end{equation}

\begin{equation}
  y_t = S_t \cdot T_t \cdot R_t
  \label{eq:multiplicativo}
\end{equation}

La descomposición STL (\emph{Sesonal and Trend decomposition using Loess}) \cite{cleveland1990stl} es uno de los métodos más conocidos para realizar esta decomposición, donde solo necesita que se le pase la frecuencia de la componente estacional para realizar el método.

\endinput

\include{conceptos-previos/metricas}

\setpartpreamble[c][0.75\linewidth]{
	\bigskip % Deja un espacio vertical en la parte superior
  Estudiamos una nueva heurística alternativa a la selección clásica de modelos: \emph{Perturbation Validation} ($PV$).
  Este método intenta medir si la función aprendida por un modelo se ajusta correctamente a la función que se está intentando aprender, intentando ser un \textbf{método complementario} a la selección en base a la validación clásica de usar particiones entrenamiento/test junto a la validación cruzada. Este método nuevo es bastante útil para la \textbf{búsqueda de hiperparámetros} del modelo, donde la validación clásica a veces no es capaz de discernir entre varios modelos con un rendimiento parecido, pero que alguno puede estar sobreajustando mucho más los datos.

  En \autoref{ch:pv-introduccion} se introduce el contexto bajo el que surge el $PV$ y que se va a intentar obtener, definiéndolo formalmente y describiendo su implementación en \autoref{ch:pv}. Finalmente, realizamos los experimentos para comprobar su utilidad, junto con los resultados y análisis de estos en \autoref{ch:pv-experimentacion}.
}
\part{Selección de modelos}\label{part:pv}
\include{capitulos/seleccion-modelos}

\setpartpreamble[c][0.75\linewidth]{
	\bigskip % Deja un espacio vertical en la parte superior
  Modelamos un \textbf{detector de anomalías} para series temporales basado en arquitectura LSTM que sea fácilmente desplegable en diversos dominios de los que se quiera reconocer patrones normales y detectar los anómalos. Además proporcionamos \textbf{métodos de alteraciones} de series temporales para poder crear muestras con las que poder realizar la validación del modelo, así como también para poder crear \emph{datasets} con datos anómalos que suplan en cierta medida la falta de estos. Usaremos las propias alteraciones para validar nuestro detector y comprobar: el buen rendimiento del modelo y la gran utilidad de las alteraciones para validarlo.

  En \autoref{ch:ad-introduccion} introducimos los problemas actuales en el dominio de detección de anomalías, donde aportamos un sistema de detección de anomalías LSTM fácil de usar y extender a muchos dominios en \autoref{ch:ad-detector} y también los métodos de alteración de series para transformarlas en anómalas en \autoref{ch:ad-alteraciones}. Probamos experimentalmente ambos resultados mediante dos experimentos: uno en un \emph{dataset} real y otro artificial, comentando los resultados y analizándolos en \autoref{ch:ad-experimentacion}.
}
\cleardoublepage\part{Detección de Anomalías}\label{part:ad}
\include{capitulos/deteccion-anomalias}


\setpartpreamble[c][0.75\linewidth]{
	\bigskip % Deja un espacio vertical en la parte superior
  Realizamos una pequeña investigación en la \textbf{predicción de series discretas}, área en la cual se han realizado muy pocos progresos. Para ello utilizaremos distintas \textbf{técnicas de discretización} para transformar los \emph{datasets} de series continuas que tenemos en discretas, estudiando su funcionamiento y uso. Después desarrollamos dos predictores basados en redes: uno normal básico y otro LSTM, de esta manera los compararemos y veremos el rendimiento entre ambos modelos.

  Primero en \autoref{ch:sd-introduccion} realizamos una breve introducción al problema proporcionando un poco de contexto. Después en \autoref{ch:discretizacion} exponemos y explicamos el uso de distintas técnicas de discretización más conocidas para transformar series continuas en discretas. Finalmente en \autoref{ch:sd-experimentacion} realizamos una pequeña experimentación para ver el rendimiento de dos modelos neuronales, base y LSTM, en cuatro series discretas obtenidas usando una par de las técnicas ya comentadas.
}
\cleardoublepage\part{Predicción de Series Temporales Discretas}\label{part:sd}
% !TeX root = ../libro.tex
% !TeX encoding = utf8

\chapter{Introducción}\label{ch:sd-introduccion}


\chapter{Discretización}\label{ch:discretizacion}

Los métodos de \textbf{discretización} de series temporales consisten en la transformación de una serie cualquiera que toma valores continuos (en $\R$ generalmente) en otra donde los valores son discretos (por ejemplo un subconjunto de $\N$) (\autoref{def:discretizacion}).

\begin{definicion}[Discretización]
  Sea una serie temporal continua $\{x_t\}_{t = 1}^n \subset \R$, un método de discretización transforma la serie $\{x_t\}$ en otra serie continua $\{y_t\}_{t = 1}^m \subset \N$ con $m \leq n$.
  \label{def:discretizacion}
\end{definicion}

En \cite{chaudhari2014discretization} se listan los principales métodos de discretización, que están clasificados según si el método es supervisado o no. Nos centraremos en los no supervisados pues no tenemos ninguna etiqueta sobre como deberían transformarse las series continuas en discretas.

\section{Discretización con mismo grosor}

La \textbf{discretización con mismo grosor} (\emph{equal width discretizacion}, EWD) consiste en la división del intervalo de los valores observados de la serie temporal en $k$ subintervalos, llamados \textbf{contenedores}, de mismo tamaño; siendo $k$ un hiperparámetro proporcionado por el usuario (\autoref{def:ewd}).

\begin{definicion}[Discretización con mismo grosor]
  Sea una serie temporal continua $\{x_t\}_{t = 1}^n \subset [a, b]$, el método de discretización con mismo grosor en $k$ contenedores transforma la serie $\{x_t\}$ en una serie discreta $\{y_t\}_{t = 1}^n$ definida de la siguiente manera para $t = 1, \ldots n$:

  $$y_t = i-1, \; \text{ tal que } \; x_t \in \left[a + (i-1) \dfrac{b - a}{k}, a + i \dfrac{b-a}{k}\right).$$
  \label{def:ewd}
\end{definicion}

Se divide el intervalo de valores $[a, b]$ en $k$ intervalos de mismo \emph{grosor} y a cada uno se le asigna un natural. A todos los valores de cada contenedor se le asigna ese natural, obteniendo la serie discretizada. La mayor desventaja de este método se presenta para series donde la \textbf{distribución} de los datos no es \textbf{uniforme} en el rango de los valores que toma.

En \autoref{fig:ewd} mostramos un ejemplo de una serie continua discretizada en 3 contenedores con mismo grosor, añadiendo los límites de los contenedores.

\begin{figure}[htpb]
  \centering
  %\hspace*{-2.5cm}
  \includegraphics[width=1\textwidth]{ewd}
  \caption{Serie continua con 3 intervalos de mismo grosor (izquierda) y la serie discretizada (derecha).}
  \label{fig:ewd}
\end{figure}

Se utiliza la implementación de \emph{sklearn} en \emph{Python} con la clase \emph{KBinsDiscretizer}, usando el parámetro \emph{strategy = uniform}.

\section{Discretización con misma frecuencia}

La \textbf{discretización con misma frecuencia} realiza una idea parecida a la de mismo grosor pero en vez de dividir el rango de valores de la serie en $k$ contenedores se considera que cada contenedor tenga el mismo número de valores, es decir, la misma \textbf{frecuencia} (\autoref{def:efd}).

\begin{definicion}[Discretización con misma frecuencia]
  Sea una serie temporal continua $\{x_t\}_{t = 1}^n \subset I$  siendo $I$ un intervalo, el método de discretización con misma frecuencia en $k$ contenedores transforma la serie $\{x_t\}$ en una serie discreta $\{y_t\}_{t = 1}^n$ definida para $t = 1, \ldots n$, como:

  $$y_t = i-1, \; \text{ tal que } \; x_t \in I_i,$$

  donde $I = \cup_{i = 1}^k I_i$ y $\ell(I_i) = \frac{n}{k}, \, i = 1, \ldots, k$.
  \label{def:efd}
\end{definicion}

Creamos los contenedores realizando una partición del rango de valores de manera que cada uno tiene la misma frecuencia, o lo que es lo mismo, cada uno tiene $\frac{n}{k}$ observaciones.

En \autoref{fig:efd} mostramos un ejemplo de una serie continua discretizada en 3 contenedores con misma frecuencia, añadiendo los límites de los contenedores.

\begin{figure}[htpb]
  \centering
  %\hspace*{-2.5cm}
  \includegraphics[width=1\textwidth]{efd}
  \caption{Serie continua con 3 intervalos de misma frecuencia (izquierda) y la serie discretizada (derecha).}
  \label{fig:efd}
\end{figure}

Se utiliza la implementación de \emph{sklearn} en \emph{Python} con la clase \emph{KBinsDiscretizer}, usando el parámetro \emph{strategy = quantile}.

\section{$k$-medias}

El algoritmo de $k$-medias (\emph{$k$-means}) \cite{macqueen1967some} es un algoritmo muy famoso y utilizado como método no supervisado de agrupamiento (\emph{clustering}) para poder dividir un conjunto de datos genérico en $k$ grupos donde se espera que cada grupo (\emph{cluster}) comparte unas características comunes.

Cada grupo está representado por un \textbf{centroide} de manera que a cada dato se le asigna al grupo cuyo centroide esté más cerca respecto la distancia euclidea (\autoref{def:k-means}).

\begin{definicion}[$k$-medias]
  Sea la matriz de datos $X = \begin{bmatrix} \textbf{x}_1 & \cdots & \textbf{x}_m \end{bmatrix}^T$ donde $\textbf{x}_i \in \R^n, \, i = 1, \ldots m$. El objetivo del algoritmo de $k$-medias es particionar los $m$ datos en $k$ grupos determinados por sus centroides $\textbf{C} = \{\pmb{\mu}_1, \ldots, \pmb{\mu}_k\}$ de manera que se minimize la inercia definida por

  $$\sum \limits^m_{i = 0} \min \limits_{\pmb{\mu} \in C} \left(||\textbf{x}_i - \pmb\mu ||^2 \right).$$
  \label{def:k-means}
\end{definicion}

En \autoref{fig:kmeans} \cite{ruiz2018kmeans} mostramos un ejemplo con datos con dos características (2D) aplicando $k$-medias con $k = 3$. Observamos que se obtienen 3 grupos con características similares.

\begin{figure}[htpb]
  \centering
  %\hspace*{-2.5cm}
  \includegraphics[width=1\textwidth]{kmeans}
  \caption{Ejemplo del algoritmo $k$-medias con $k=3$ aplicado a unos datos con 2 características (izquierda) del que se obtienen 3 grupos (derecha).}
  \label{fig:kmeans}
\end{figure}

Con este algoritmo obtenemos los $k$ contenedores como si fuesen los grupos representados por los centroides. Por tanto, cada punto se asigna al contenedor cuyo centroide esté más cerca (\autoref{def:kd}).

\begin{definicion}[Discretización con $k$-medias]
  Sea una serie temporal continua $\{x_t\}_{t = 1}^n$, el método de discretización con $k$-medias transforma la serie $\{x_t\}$ en una serie discreta $\{y_t\}_{t = 1}^n$ definida para $t = 1, \ldots, n$ como:

  $$y_t = i-1 \in N, \; \text{ tal que } \; i = \argmin \limits_{j = 1, \ldots, k} ||x_t - \pmb\mu_j||,$$

  donde $\pmb \mu_j$, $j = 1, \ldots, k$ son los centroides obtenidos en el algoritmo de $k$-medias.
  \label{def:kd}
\end{definicion}

Este método es más interesante que los anteriores ya que agrupa mediante las características de los datos por lo que puede obtenerse una discretización más significativa.

En \autoref{fig:dk} mostramos un ejemplo de una serie continua discretizada en 3 grupos, añadiendo los límites de los contenedores.

\begin{figure}[htpb]
  \centering
  %\hspace*{-2.5cm}
  \includegraphics[width=1\textwidth]{dk}
  \caption{Serie continua con 3 intervalos de $k$-medias (izquierda) y la serie discretizada (derecha).}
  \label{fig:dk}
\end{figure}


Se utiliza la implementación de \emph{sklearn} en \emph{Python} con la clase \emph{KBinsDiscretizer}, usando el parámetro \emph{strategy = kmeans}.

\section{SAX}

El último método que desarrollamos es \emph{Symbolic Aggregate Aproximation} (SAX) \cite{lin2007experiencing}, un método más avanzado diseñado específicamente para series temporales. Este método transforma las series temporales en palabras de longitud igual o menor que las originales, permitiendo una discretización simbólica y una reducción de la dimensión.

Se necesitan indicar el número $k$ de contenedores que se codifican como una palabra de un alfabeto de tamaño $k$ (tomamos las $k$ primeras letras del abecedario latino) y el tamaño de ventana $w$ para dividir la serie en $\frac{n}{w}$ ventanas.

El algoritmo tiene tres pasos fundamentales: primero se toma la serie original y se normaliza (media 0 y varianza 1) (\autoref{def:normalizacion}).

\begin{definicion}[Normalización]
  Sea una serie temporal $\{x_t\}_{t = 1}^n$, su serie normalizada $\{s_t\}_{t = 1}^n$ está definida de la siguiente forma para $t = 1, \ldots, n$:

  $$s_t = \dfrac{x_t - \overline{x}}{\sigma_x},$$

  donde $\overline{x}$ es le media de la serie y $\sigma_x$ la desviación típica de la serie.
  \label{def:normalizacion}
\end{definicion}

Después se aplica a la serie normalizada una reducción de dimensión mediante el método PAA (\emph{Piecewise Aggregate Approximation}) \cite{keogh2001dimensionality} (\autoref{def:paa}).

\begin{definicion}[PAA]
  Sea una serie temporal $\{x_t\}_{t = 1}^n$ y un tamaño de ventana $w$, el método PAA reduce la dimensión de la serie transformándola en una serie de longitud $m = \frac{n}{w}$ denotado $\{\overline{x}_t\}_{t = 1}^m$ determinado por la siguiente expresión para $t = 1, \ldots, m$:

  $$\overline{x}_t = \dfrac{1}{n} \sum \limits_{i = (t-1)w + 1}^{tw} x_i.$$
  \label{def:paa}
\end{definicion}

Este método reduce la dimensión de la serie dividiendo en $m$ ventanas la serie normalizada y tomando la media de cada uno de ellos. En \autoref{fig:paa} mostramos un ejemplo de PAA con un tamaño de ventana $w$ de 157.

\begin{figure}[htpb]
  \centering
  %\hspace*{-2.5cm}
  \includegraphics[width=.55\textwidth]{paa}
  \caption{Método PAA aplicado a una serie temporal con $w = 157$.}
  \label{fig:paa}
\end{figure}

Finalmente, aplicamos una discretización de la serie obtenida del método PAA asignando a cada ventana una letra de un alfabeto siendo generalmente el subalfabeto formado por los $k$ primeras letras del alfabeto latino.

La asignación de las letras se realiza de una manera similar a las técnicas descritas anteriormente, dividiendo el rango de valores en contenedores a los que se asigna una letra para cada uno. En este caso se divide la función de densidad de una distribución normal de manera que cada contenedor tiene el mismo área bajo la curva (\autoref{def:sax}).

\begin{definicion}[SAX]
  Sea una serie temporal $\{x_t\}_{t = 1}^n$ y un alfabeto de tamaño $k$ denotado por $A = \{a_1, \ldots, a_k\}$ el método SAX asigna a la serie una palabra $w = \alpha_1\alpha_2\ldots\alpha_n$, $\alpha_t \in A, \, t = 1, \ldots, n$ de la siguiente manera para $t = 1, \ldots n$:

  $$\alpha_t = a_i, \; \text{ si } \; \beta_{i-1} \leq x_t < \beta_i.$$

  donde $\beta_i$, $i = 1, \ldots n$, llamados puntos de ruptura (\emph{breakpoints}), son los valores tales que el área debajo de la curva de una distribución normal $N(0,1)$ desde $\beta_i$ hasta $\beta_{i+1}$ es $\frac{1}{k}$, $i = 1, \ldots n-1$.
  \label{def:sax}
\end{definicion}

Utilizando la función inversa de distribución acumulada podemos obtener fácilmente estos puntos de ruptura para cualquier $k$. En \autoref{fig:sax} \cite{lin2007experiencing} mostramos un ejemplo de como SAX transforma una serie temporal continua en la palabra \textbf{baabccbc}.

\begin{figure}[htpb]
  \centering
  %\hspace*{-2.5cm}
  \includegraphics[width=.85\textwidth]{sax}
  \caption{Método SAX aplicado a una serie temporal con $w = 16$ y $k = 3$. Se aplica el método PAA y a cada sección se le asigna una letra.}
  \label{fig:sax}
\end{figure}

Se ha implementado la clase \emph{SAX} en \emph{Python} que ejecuta el método SAX y PAA en una versión basada para usar con la biblioteca \emph{sklearn}.

Cabe mencionar que aunque SAX transforme la serie en una palabra, podemos codificar cada letra como un número entero y poder trabajar con valores numéricos. Para ello se ha implementado también la clase \emph{DiscretizationEncoder} que usado junto con la clase \emph{SAX} nos permite hacer la codificación de las letras a numeros enteros.

\chapter{Experimentación}\label{ch:sd-experimentacion}


\include{capitulos/conclusiones-trabajo}

% --------------------------------------------------------------------
% APPENDIX: Opcional
% --------------------------------------------------------------------

\appendix % Reinicia la numeración de los capítulos y usa letras para numerarlos
\pdfbookmark[-1]{Apéndices}{appendix} % Alternativamente podemos agrupar los apéndices con un nuevo \part{Apéndices}

\include{apendices/documentacion}
% !TeX root = ../libro.tex
% !TeX encoding = utf8

\chapter{Software}\label{ap:software}

\section{Archivos del proyecto}

Encontramos el \href{https://github.com/MiguelLentisco/tfg}{proyecto} en la plataforma \emph{Github} , con la siguiente estructura:

\begin{itemize}
  \item $PV$: Parte selección de modelos
    \begin{itemize}
      \item $resultados$: contiene los resultados del experimento en .csv y .png para TS Y CMFTS.
      \item $src$: archivos fuente.
      \begin{itemize}
        \item $PV.py$: contiene la clase PV.
        \item $KNN.py$: contiene la clase KNN.
        \item $LSTM.py$: contiene la clase LSTM.
        \item $RClassifiers.py$: contiene las clases DTW, C45, C50, CPart y RPart.
        \item $Utils.py$: funciones auxiliares.
        \item $experimento\_PV.py$: experimento para selección de modelos.
        \item $experimento\_hiper.py$: experimento para los hiperparámetros.
      \end{itemize}
    \end{itemize}
  \item $AD$: Parte detección de anomalías
    \begin{itemize}
      \item $models$: contiene los pesos de los modelos obtenidos.
      \item $src$: archivos fuente.
        \begin{itemize}
          \item $alteraciones.py$: métodos para alterar series.
          \item $calc_pr.py$: métodos para calcular la métrica PR.
          \item $detector.py$: detector LSTM.
          \item $experimento\_AD.py$: experimento anomalías.
        \end{itemize}
    \end{itemize}
  \item $doc$: archivos referentes a la memoria del proyecto.
  \item $Datasets$: \emph{datsets} con las versiones TS y CMFTS.

\end{itemize}

\section{Lenguajes utilizados}

La mayoría de código ha sido implementado en \emph{Python}, habiéndose implementado alguna función y usando paquetes de \emph{R}.

\section{Librerías utilizadas}

Listamos las librerías utilizadas en el proyecto. En \emph{Python}:

\begin{itemize}
  \item \emph{matplotlib}: para imprimir gráficos.
  \item \emph{numpy}: para trabajar con arrays, matrices, tensores.
  \item \emph{pandas}: cargar y guardar \emph{datasets}.
  \item \emph{keras}: utilizada para implementar modelos de redes neuronales LSTM.
  \item \emph{scikit-learn}: se han usado distintos clasificadores, métricas y preprocesado de \emph{datasets}.
  \item \emph{scipy}: para señales gaussianas, pulso sinusoidal-gaussiano y estimación de distribución.
  \item \emph{statsmodels}: descomposición STL.
  \item \emph{rpy2}: interfaz para utilizar \emph{R} con \emph{Python}.
\end{itemize}

En \emph{R}:

\begin{itemize}
  \item \emph{RWeka}: árbol de decisión C4.5
  \item \emph{C50}: árbol de decisión C5.0
  \item \emph{rpart}: árbol de decisión RPart.
  \item \emph{partykit}: árbol de decisión CTree.
  \item \emph{IncDTW}: función para calcular la métrica DTW.
\end{itemize}

\endinput

% Añadir tantos apéndices como sea necesario

% --------------------------------------------------------------------
% GLOSARIO: Opcional
% --------------------------------------------------------------------

%\include{glosario}


% -------------------------------------------------------------------
% BACKMATTER
% -------------------------------------------------------------------

\backmatter % Desactiva la numeración de los capítulos
\pdfbookmark[-1]{Referencias}{BM-Referencias}

% BIBLIOGRAFÍA
%-------------------------------------------------------------------

\setbibpreamble{Las referencias se listan por orden alfabético. Aquellas referencias con más de un autor están ordenadas de acuerdo con el primer autor.\par\bigskip}
\bibliographystyle{alphaurl}
\begin{small} % Normalmente la bibliografía se imprime en un tamaño de letra más pequeño.
\bibliography{library.bib}
\end{small}


% ÍNDICE TERMINOLÓGICO  (Opcional)
%-------------------------------------------------------------------

\cleardoublepage
\begin{footnotesize} % Normalmente el índice se imprime en un tamaño de letra más pequeño.
\printindex
\end{footnotesize}
% !TeX root = ../libro.tex
% !TeX encoding = utf8

%*******************************************************
% Agradecimientos
%*******************************************************

\chapter*{Agradecimientos}

Agradezco a mis tutores José Manuel Benítez y Miguel Lastra por ofrecerme este proyecto y haber estado ayudándome con sus consejos, explicaciones y pautas que han sido esenciales para desarrollar este trabajo y también para acercarme al mundo de la investigación.

También a mi novio Víctor López y a su familia por su apoyo, cariño y comprensión en las circunstancias tan imprevisibles que hemos estado viviendo en estos meses pasados. De igual manera a mi madre Ana Ballesteros y a mi padre Antonio Lentisco que no han podido estar mucho físicamente pero siempre han confiado en mi.

A todos mis amigos y compañeros de carrera que han estado conmigo desde hace ya tantos años, apoyándome y ayudándome en cada tramo del camino. En especial gracias a Sofía Almeida, Pablo Baeyens, Antonio Coín, José María Martín, Antonio Martín, Laura Gómez, Daniel Pozo y Javier Saez que me han ayudado a corregir y pulir el trabajo y también han estado escuchándome y apoyándome a lo largo de todo el proyecto.

\endinput

\end{document}
