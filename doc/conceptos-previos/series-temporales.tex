% !TeX root = ../libro.tex
% !TeX encoding = utf8
%
%*******************************************************
% Series Temporales
%*******************************************************

\chapter{Series Temporales}\label{ch:st}

[EN CONSTRUCCIÓN]

\subsection{Descomposición STL}

En las series se pueden observar ciertos componentes comunes con patrones diferenciados en los que se pueden descomponer para hacer un mejor análisis y manuplacion de estas \cite{hyndman2018forecasting}:

\begin{enumerate}
  \item \textbf{Tendencia}: patrones de crecimiento o disminución a largo plazo.
  \item \textbf{Estacionalidad}: patrones afectados por factores estacionales (un día concreto del año, una estación como verano) que suelen tener una frecuencia fija.
  \item \textbf{Ciclo}: fluctuaciones que no parece tener una frecuencia fija, es decir, que son efímeros.
\end{enumerate}

Estas descomposiciones suelen realizarse en las componentes mencionadas: la \textbf{tendencia-cíclica} $S_t$ (se suele decir solamente tendencia), la \textbf{estacionalidad} $T_t$ y los \textbf{restos} $R_t$ (lo que sobra). Además podemos considerar un modelo aditivo \eqref{eq:aditivo} donde se suman las componentes o multiplicativo \eqref{eq:multiplicativo}; donde se multiplican \cite{hyndman2018forecasting}.

\begin{equation}
  y_t = S_t + T_t + R_t
  \label{eq:aditivo}
\end{equation}

\begin{equation}
  y_t = S_t \cdot T_t \cdot R_t
  \label{eq:multiplicativo}
\end{equation}

La descomposición STL (\emph{Sesonal and Trend decomposition using Loess}) \cite{cleveland1990stl} es uno de los métodos más conocidos para realizar esta decomposición, donde solo necesita que se le pase la frecuencia de la componente estacional para realizar el método.

\endinput
